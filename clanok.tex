% Metódy inžinierskej práce

\documentclass[10pt,twoside,slovak,a4paper]{article}

\usepackage[slovak]{babel}
%\usepackage[T1]{fontenc}
\usepackage[IL2]{fontenc} % lepšia sadzba písmena Ľ než v T1
\usepackage[utf8]{inputenc}
\usepackage{graphicx}
\usepackage{url} % príkaz \url na formátovanie URL
\usepackage{hyperref} % odkazy v texte budú aktívne (pri niektorých triedach dokumentov spôsobuje posun textu)

\usepackage{cite}
%\usepackage{times}

\pagestyle{headings}

\title{Algoritmy odporúčania filmov v Netflix\thanks{Semestrálny projekt v predmete Metódy inžinierskej práce, ak. rok 2024/25, vedenie: Ing. Richard Marko, PhD}}

\author{Yehor Bohuslavskyi\\[2pt]
	{\small Slovenská technická univerzita v Bratislave}\\
	{\small Fakulta informatiky a informačných technológií}\\
	{\small \texttt{xbohuslavkyi@stuba.sk}}
	}

\date{\small 3. oktober 2024} % upravte



\begin{document}

\maketitle

\begin{abstract}
úto tému som si vybral, pretože je veľmi zaujímavé, ako technológie a umelá inteligencia(ktorá je v súčasnosti jednou z najrozvíjajúcejších sa tém) ovplyvňujú naše každodenné rozhodovanie o tom, čo sledujeme. Ja som zvedavý, že ako po zhliadnutí jedného tureckého filmu, ktorý som ani neohodnotil, sa mi v odporúčaniach začali objavovať podobné filmy. 

V systéme sa nachádza viac, ako 15 000 filmov a zobrazuje vám len tie, ktoré sa vám budú páčiť (neskôr dozvieme sa, ako to funguje). Fakt: ani jeden používateľ Netflix nebol by schopný si samostatne nájsť film alebo seriál, ktorý by sa mu páčil bez požívania algoritmu odporúčania filmov [1].
 
V tomto článku sa podrobne pozrieme na algoritmy spoločnosti Netflix. V súčasnosti je systém Netflix postavený na algoritme, ktorý využíva umelú inteligenciu a strojové učenie. Pozrieme sa aj na umiestnenie odporúčaní na obrazovke, kde je niekoľko typov návrhov. Prvým je „Odporúčané pre vás“, po ňom idú „Trendy“, na stránke sa nachádza aj záložka „New Releases“.

Podrobne sa pozrieme aj na filtračné algoritmy, ktoré sa delia na dva hlavné typy. Akú rolu odohrávajú hodnotenia a správania používateľov v rámci algoritmu collaborative filtering a „filtrovanie na základe obsahu“. 

Zaujímavým aspektom sú aj systémy hodnotenia. Patrí medzi ne „Personalizované hodnotenie videa (PVR)“, „Top-N Ranking“, „Rebríček obľúbených filmov“ a „Zoznam zaujímavého obsahu na neskoršie sledovanie“. Každé z týchto hodnotení je personalizované pre jednotlivých používateľov, čo je veľmi zaujímavé.

V rámci článku sa budeme venovať aj typom údajov, ktoré Netflix používa. Tie sa zhruba delia na dva typy: základné typy zhromažďovaných údajov a dodatočné informácie. Medzi základné patria „interakcie používateľa so stránkou“, ktoré pozostávajú z histórie sledovania a hodnotení. Druhým typom sú „korelačné údaje“, „informácie o obsahu knižnice Netflix“, ako napríklad žánre. Ďalšie uvidíme ako ovplyvňujú  špecifickejšie typy informácií, ako sú napríklad denný čas, používané zariadenie alebo priemerná dĺžka sledovania.

\end{abstract}



\section{Úvod}

Motivujte čitateľa a vysvetlite, o čom píšete. Úvod sa väčšinou nedelí na časti.

Uveďte explicitne štruktúru článku. Tu je nejaký príklad.
Základný problém, ktorý bol naznačený v úvode, je podrobnejšie vysvetlený v časti~\ref{nejaka}.
Dôležité súvislosti sú uvedené v častiach~\ref{dolezita} a~\ref{dolezitejsia}.
Záverečné poznámky prináša časť~\ref{zaver}.



\section{Nejaká časť} \label{nejaka}

Z obr.~\ref{f:rozhod} je všetko jasné. 

\begin{figure*}[tbh]
\centering
%\includegraphics[scale=1.0]{diagram.pdf}
Aj text môže byť prezentovaný ako obrázok. Stane sa z neho označný plávajúci objekt. Po vytvorení diagramu zrušte znak \texttt{\%} pred príkazom \verb|\includegraphics| označte tento riadok ako komentár (tiež pomocou znaku \texttt{\%}).
\caption{Rozhodujúci argument.}
\label{f:rozhod}
\end{figure*}



\section{Iná časť} \label{ina}

Základným problémom je teda\ldots{} Najprv sa pozrieme na nejaké vysvetlenie (časť~\ref{ina:nejake}), a potom na ešte nejaké (časť~\ref{ina:nejake}).\footnote{Niekedy môžete potrebovať aj poznámku pod čiarou.}

Môže sa zdať, že problém vlastne nejestvuje\cite{Coplien:MPD}, ale bolo dokázané, že to tak nie je~\cite{Czarnecki:Staged, Czarnecki:Progress}. Napriek tomu, aj dnes na webe narazíme na všelijaké pochybné názory\cite{PLP-Framework}. Dôležité veci možno \emph{zdôrazniť kurzívou}.


\subsection{Nejaké vysvetlenie} \label{ina:nejake}

Niekedy treba uviesť zoznam:

\begin{itemize}
\item jedna vec
\item druhá vec
	\begin{itemize}
	\item x
	\item y
	\end{itemize}
\end{itemize}

Ten istý zoznam, len číslovaný:

\begin{enumerate}
\item jedna vec
\item druhá vec
	\begin{enumerate}
	\item x
	\item y
	\end{enumerate}
\end{enumerate}


\subsection{Ešte nejaké vysvetlenie} \label{ina:este}

\paragraph{Veľmi dôležitá poznámka.}
Niekedy je potrebné nadpisom označiť odsek. Text pokračuje hneď za nadpisom.



\section{Dôležitá časť} \label{dolezita}




\section{Ešte dôležitejšia časť} \label{dolezitejsia}




\section{Záver} \label{zaver} % prípadne iný variant názvu



%\acknowledgement{Ak niekomu chcete poďakovať\ldots}


% týmto sa generuje zoznam literatúry z obsahu súboru literatura.bib podľa toho, na čo sa v článku odkazujete
\bibliography{literatura}
\bibliographystyle{plain} % prípadne alpha, abbrv alebo hociktorý iný
\end{document}
